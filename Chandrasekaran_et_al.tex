\documentclass[letterpaper]{article}
\usepackage{aaai17}
\usepackage{times}
\usepackage{helvet}
\usepackage{courier}
\usepackage{url}
\usepackage{latexsym}
\usepackage{xparse}

\newsavebox{\fminipagebox}
\NewDocumentEnvironment{fminipage}{m O{\fboxsep}}
 {\par\kern#2\noindent\begin{lrbox}{\fminipagebox}
  \begin{minipage}{#1}\ignorespaces}
 {\end{minipage}\end{lrbox}%
  \makebox[#1]{%
    \kern\dimexpr-\fboxsep-\fboxrule\relax
    \fbox{\usebox{\fminipagebox}}%
    \kern\dimexpr-\fboxsep-\fboxrule\relax
  }\par\kern#2
 }
 
\frenchspacing
\setlength{\pdfpagewidth}{8.5in}
\setlength{\pdfpageheight}{11in}
\pdfinfo{
/Title (Using Discourse Signals for Robust Instructor Intervention Prediction)
/Author (Muthu Kumar Chandrasekaran, Carrie Demmans Epp, Min-Yen Kan, Diane Litman)
/keywords (MOOC discussion forum, MOOC, Instructor intervention, PDTB discourse relations)}

\setcounter{secnumdepth}{0}  
 \begin{document}
\title{Using Discourse Signals for Robust Instructor Intervention Prediction}
\author{Muthu Kumar Chandrasekaran$^1$, Carrie Demmans Epp$^{2,3}$, 
		 Min-Yen Kan$^{1,4}$, Diane Litman$^{2,5}$ \\ 
$^1$ Department of Computer Science, School of Computing, National University of Singapore\\
$^2$ Learning Research and Development Center, University of Pittsburgh\\
$^3$ University Center for Teaching and Learning, University of Pittsburgh\\
$^4$ Interactive and Digital Media Institute, National University of Singapore, Singapore\\
$^5$ Department of Computer Science, University of Pittsburgh\\
\{muthu.chandra, kanmy\}@comp.nus.edu.sg\hspace{1cm}\{cdemmans,dlitman\}@pitt.edu
}

\maketitle
\begin{abstract}
We tackle the prediction of instructor intervention in student posts 
from discussion forums 
in Massive Open Online Courses (MOOCs). Our key finding is 
that using automatically obtained discourse relations improves the 
prediction of when instructors intervene in student discussions, when 
compared with a state-of-the-art, feature-rich baseline. Our supervised 
classifier makes use of an automatic discourse parser which outputs Penn 
Discourse Treebank (PDTB) tags that represent in-post discourse features. 
We show PDTB relation-based features increase the robustness of the 
classifier and complement baseline 
features in recalling more diverse instructor intervention patterns. 
In comprehensive experiments over 14 MOOC offerings from  
several disciplines, the PDTB discourse features improve performance 
on average. The resultant models are less dependent on domain-specific 
vocabulary, allowing them to better generalize to new courses.
\end{abstract}

\section{Introduction}
\label{sect:intro}
Massive Open Online Courses (MOOCs) aim to scale learning by creating 
virtual classrooms that eliminate the need for students to be co-located 
with instructional staff and each other. To facilitate interaction, MOOC 
platforms have discussion forums where students can interact with instructional 
staff -- hereafter called {\it instructors} -- and their classmates. Forums are 
typically the only mode of interaction between instructors and 
students. Forums often contain hundreds of posts from several thousand 
students, each post competing with others for instructor attention.  
Reading and responding to student queries in forums is an essential teaching 
activity that helps instructors gauge student understanding of course 
content. Intervention is argued to facilitate student learning where an 
instructor's presence 
and intervention in student discussions improves learning 
outcomes in MOOCs~\cite{chen2016} and other online learning 
environments~\cite{garrison_critical_1999,phirangee_exploring_2016}. 
However, instructors need to be selective when answering student posts due to 
their limited bandwidth. One selection strategy is to respond 
to posts that will maximally benefit the most students in a course. 
Along these lines, \citeauthor{chandrasekaran2015towards} (\citeyear{chandrasekaran2015towards}) 
proposed an intervention taxonomy based on transactive discourse that details 
the situations in which certain types of interventions would maximally benefit 
students. 

Consistent with this taxonomy, \citeauthor{chandrasekaran2015learning} 
showed that intervention strategies in MOOC forums differ widely.
The factors behind different instructor intervention strategies include the 
instructors' pedagogical philosophy, a desire to encourage learner 
interaction, a desire to ensure students understand course content, and a 
need to correct misconceptions~\cite{phirangee_fill_2016}. 
The intervention strategy chosen was found to impact student learning 
significantly~\cite{mazzolini2003,mazzolini2007}.

\begin{figure}
\small 
\begin{tabular}{|p{7.8cm}|}
\hline 

\textbf{Student 1 (original poster)}: Hie guys I m sorry
\textbf{if}$_{Cont}$ my question is naive in
anyway. \textbf{But}$_{Comp}$ I am confused ...  Say suppose,
\textbf{if}$_{Cont}$ we were to take the 5-6 Descending
progression...  \textbf{and so on}$_{Exp}$ I can’t help
\textbf{but}$_{Comp}$ see the ...  \textbf{Now}$_{Temp}$ {\bf
if}$_{Cont}$ I need to apply the same progression to a minor
scale, \textbf{then}$_{Cont}$ should I ... In the case of circle
of fifths progression, \textbf{if}$_{Cont}$
I... \textbf{So}$_{Cont}$ we apply VII major instead?... \\

\hline \\

\textbf{Student 2 (1st reply)}: In a minor key the chords ... are
\textbf{as follows}$_{Exp}$..., \textbf{but}$_{Cont}$ we ...,
\textbf{because}$_{Cont}$ dominant chords should be... The
wrinkle in minor keys is that to create the V chord, you have to
raise the seventh degree of the scale (\textbf{so}$_{Cont}$ in a
minor you sharp the g, \textbf{when$_{Temp}$} it occurs in the V
chord). I am not sure, \textbf{but}$_{Exp}$ I think ... I believe
you can use that chord as an substitute for the V chord in a
minor key, \textbf{just as}$_{Comp}$ you can in the major key,
\textbf{but}$_{Exp}$ I'd depend on the staff to confirm
that. ...Take this with a grain of salt, \textbf{as}$_{Cont}$ I
am learning, too, \textbf{but}$_{Exp}$ I think it is correct. \\

\hline \\

\textbf{Instructor's reply}: Hi [Student1] [Student2] is heading towards 
the right direction. He is right about the circle of fifths ...  \\

\hline
\end{tabular}
\caption{An example from our {\sc Classical-1} MOOC in our corpus
  where student confusion could benefit from instructor intervention.
  Here, discourse connectives are in bold and annotated with their
  Level-1 PDTB senses: (Temp)oral, (Cont)ingency, (Comp)arison, or
  (Exp)ansion.}
\vspace{-4mm}
\label{fig:example-intro}
\end{figure}

Earlier work also shows that content-based features, which
include simple linguistic features derived from student vocabulary 
(e.g., word unigrams), 
signal common traits that are useful for predicting instructor 
interventions~\cite{chandrasekaran2015learning,chaturvedi2014predicting}. 
However, a key problem with surface-level vocabulary features is that they 
vary widely across courses, as courses from different subject areas use 
different domain-specific vocabulary. Predictive models trained on such 
word-based 
features do not generalize well when applied to new unseen courses in different 
disciplines. In contrast, function words such as conjunctions (e.g., ``and'', 
``because'') occur frequently across corpora and can be leveraged to 
create more robust features, 
such as what was done in the related task of predicting transactivity in 
educational dialogues~\cite{joshi2007}.

Our work focuses on a specific subclass of 
function words --- discourse connectives --- as they serve to 
connect clauses and signal the communicative intent of the writer. 
The Penn Discourse Treebank (PDTB; \citeauthor{Prasad2008} 
\citeyear{Prasad2008}) formalism identifies connectives 
that signal discourse relations and categorises them into senses. 
As can be seen in Figure~\ref{fig:example-intro}, both student posts 
contain a number of \textit{if...then}, \textit{but} connectives that 
belong to the contingency and comparison senses of the PDTB. It is 
common to find such patterns in student posts 
expressing confusion as they hedge and hypothesise to check their understanding,
which can call for instructor intervention. In contrast, the student post in 
Figure~\ref{fig:example-danger} is confident in tone, uses the imperative form, 
and is devoid of such connectives. These examples motivate us further to extract 
discourse-based features from student discussion forum posts. We hypothesize 
that student posts differ in their discourse structures and that some of these 
structures will attract instructor intervention. We additionally hypothesize 
discourse features will yield models for predicting instructor intervention that 
generalize well to unseen courses. 

Following prior work, we cast the problem of predicting instructor intervention
as a binary classification problem where intervened and non-intervened threads 
are treated as positive and negative instances, respectively.
We test our hypotheses extrinsically using automatically extracted discourse 
features that follow the PDTB 
formalism, enriching a state-of-the-art baseline model for predicting
instructor intervention. In contrast to prior work on single MOOC
instances, our experiments are comprehensive, covering a corpus of 14
MOOC instances from various disciplines, offered by two different
universities.

Our results show that PDTB features improve the state-of-the-art baseline 
performance by 3.4\% (Table~\ref{tab:resultOut}) when trained on a large 
out-of-domain dataset and by 0.4\% (Table~\ref{tab:resultIn}) when trained on a  
smaller in-domain dataset. Further, PDTB features on 
their own perform comparably to the state-of-the-art on select MOOC offerings. 
We show that unlike vocabulary based features, PDTB features are robust to 
domain differences across MOOCs.

\section{Related Work} 
\label{sect:related}

Predicting instructor intervention became a viable problem to
study with the availability of large amounts of educational discussion 
forum data from MOOCs. Chaturvedi et~al.~\shortcite{chaturvedi2014predicting} 
first specified the problem as predicting which MOOC discussion forum threads 
instructors would post to, where an instructor post is considered 
an intervention.

They modelled macro-level thread discourse structure (i.e.,
across posts), demonstrating that their model outperformed a
representative classifier endowed with many lexical and other
surface level features. However, later work failed to replicate their 
results across a much broader study of MOOC forums culled from several 
universities~\cite{chandrasekaran2015learning}. This work cited the 
large variety of course content and instructor preference as likely 
causes to the non-portability of the initial study's results. 
\citeauthor{chandrasekaran2015learning} also showed that other simple 
features such as sub-forum type, thread length and surface level 
linguistic cues outperform the discourse model from the earlier work.

Our work uses \citeauthor{chandrasekaran2015learning}
\shortcite{chandrasekaran2015learning}, hereafter denoted as EDM'15,
as a starting point and as a state-of-the-art baseline for comparison.
In contrast to both prior works, we model microscopic discourse
structures -- {\it i.e.}, sentence and clause-level discourse within
student posts. We also eschew vocabulary-dependent approaches, such as
those suggested by \citeauthor{ramesh2015} \shortcite{ramesh2015}
where intervention was based on emergent topics and subtopics 
from each course, since we seek models that generalize across a wide
variety of courses.

\textbf{Discourse Parsing Applications.} As forum discussions feature
dialogue and argumentation, we felt strongly that providing discourse
analyses would improve prediction performance. Automatic discourse
parsing discovers the relationship between clauses or sentences in
contiguous text. Discourse parsing usually categorizes the inferred 
relation with a discourse type.

With the availability of large-scale discourse annotations on top of
the Penn Treebank, the PDTB formalism for
discourse annotation has become a {\it de facto} standard for
automated discourse parsing and analyses. Importantly, the PDTB
formalism splits the detection of discourse relations into ones
signaled {\it explicitly} by a discourse connective (e.g., the
connective ``if'' often signals a {\it Contingency} relation between
its arguments, as in the first connective from
Figure~\ref{fig:example-intro}) from {\it implicitly} signaled ones
that have no overt connective. As a result, automatic discourse
relation identification relying solely on explicit connectives is
rather precise but provides low overall coverage.

While the PDTB annotated corpus is built largely on newswire (e.g.,
{\it Wall Street Journal}), the PDTB tag set and derived parsers have
found applicability in a variety of NLP tasks on different corpora: Li
et~al.~\shortcite{li2014} showed the influence of PDTB explicit
relations on machine translation quality.  Mih{\u{a}}il{\u{a}} and
Ananiadou~\shortcite{mihuailua2014} found causal relations using PDTB
in scientific Biomedical journals.  Importantly, PDTB's applicability
to the related form of user-generated text, especially expository
texts, has also been studied: Faulkner
et~al.~\shortcite{faulkner2014automated} used PDTB discourse features
to support argument classification in student essays from the
International Corpus of Learner
English\footnote{\url{https://www.uclouvain.be/en-cecl-icle.html}}.
Also similarly, in performing a selection problem close to ours, Wang
et~al.~\shortcite{Wang2012} studied discourse parsing's utility for
retweetability of tweets and found correlations between the discourse
type and sentiment polarity.  \citeauthor{swanson2015}
(\citeyear{swanson2015}) also found these relations to be useful in
argument extraction from general web forum text.

MOOC discussion forum text is user-generated, expository and
conversational all at once. For this reason, we hypothesize that
(explicitly marked) discourse parsing would improve the prediction of
instructor intervention. Our hypothesis extends to forums in any
online learning environment, such as the learning management systems
that schools and universities host for their students.

\section{Data and Preprocessing}
\label{sect:data}

The corpus for our experiments consists of data from 14 Coursera MOOC
offerings\footnote{As of September 2016, Coursera, a commercial MOOC
  platform: \url{https://www.coursera.org}, hosted 1157 courses in
  English spanning the humanities, social sciences, engineering, and
  sciences.} that are spread across 7 courses from the authors'
universities. The included MOOCs taught a variety of subjects spanning
the humanities and sciences. All courses relied on videos to deliver
core content. Different instructional approaches and learning
activities (e.g., peer/self-assessments, prompted discussions, tests,
or papers) were used and influenced discussion forum activities. This
variety is apparent through the number of instructional staff (i.e.,
instructor or teaching assistants) who posted in the forums.  The
varied approaches are also apparent through their intervention ratios
where {\sc Clinical-1}\footnote{``-$n$'' refers to the $n^{th}$ time
  the course was offered; ``{\sc Clinical-1}'' stands for the first
  offering of the Clinical course.} (Row~6; Table~\ref{tab:data}) had
the highest intervention ratio (0.73) and {\sc Disaster-3} (Row~10 of
Table~\ref{tab:data}) had the lowest (0.02).

Coursera forums are divided into several sub-forums. Each of the sub-forums was 
manually categorized, using the definitions  
from~\cite{chandrasekaran2015learning}, as belonging to one of the following 
types: errata, exam, lecture, homework, general, peer review, study group, or 
technical issues. Similar to prior work on intervention prediction, the general, 
study group, peer review  and technical issues sub-forums and their threads were 
removed since they are noisy and do not focus on course content 
(e.g., social discussions and reports of technical issues). As the task is 
instructor intervention prediction, we also omit threads where the first post 
was made by an instructor. Table~\ref{tab:data} shows the number of threads 
that are used in our model.

We truncate threads after the first instructor post (dropping
subsequent student posts) because predicting 
the first instructor intervention is a viable problem and distinct
from predicting subsequent, follow-up interventions that can be
motivated by different reasons.  Further, after an intervention,
discussions gain visibility which can
inflate feature counts in our prediction task. To extract features, 
we first tokenize thread text.  We replaced instances of non-lexical
references such as equations, URLs and timestamps, with the tokens:
$<$EQU$>$, $<$URL$>$, $<$TIMEREF$>$, respectively. These tokens are a
feature of the baseline prediction system (see ``Baseline (EDM'15)''
section). They also enable the discourse parser to skip unparsable
text\footnote{The discourse parser extracts syntactic and dependency
  parse features from the Stanford parser, which fails on these types
  of non-lexical strings; available at
  \url{http://nlp.stanford.edu/software}.}. Stopwords and words of
length less than 3 were removed before extracting the baseline
features. Stopwords were not removed when extracting discourse
features ({\it cf} ``Discourse Feature Extraction'' Section).
Our work examines three predictive models: 1) the baseline (EDM'15),
2) a system with only PDTB discourse relations as features (PDTB), and
3) an augmented system where discourse relations are also used (EDM'15
+ PDTB; E+P for short).

\begin{table}
\centering
\def\arraystretch{1.2}% 
\small
\begin{tabular}{|l|l|r|r|r|}
\hline 
\bf Uni. & \bf Course & \bf \# of  & \bf  \# of non--  & \bf I. Ratio \\ 
\bf &\bf (-Iteration) & \bf 	intervened	 & \bf intervened & \bf \\ 
\hline
NUS & {\sc Classic-1} & 164 & 527 & 0.31\\
& {\sc Classic-2 } & 17 & 155 & 0.11\\
& {\sc Reason-1 } & 58 & 231 & 0.25\\
& {\sc Reason-2 } & 40 & 265 & 0.15\\
\hline
Pitt & {\sc AccTalk} & 98& 254& 0.39\\
& {\sc Clinical-1} & 33& 45& 0.73\\
& {\sc Clinical-2} & 32& 82& 0.39\\
& {\sc Disaster-1} & 81& 2332& 0.03\\
& {\sc Disaster-2} & 53& 718& 0.07\\
& {\sc Disaster-3} & 18& 960& 0.02\\
& {\sc Nuclear-1} & 272& 779& 0.35\\
& {\sc Nuclear-2} & 93&255& 0.36\\
& {\sc Nutrition-1} & 98& 2346& 0.04\\
& {\sc Nutrition-2} & 73& 1475& 0.04\\
\hline
 & Total & 1,130& 10,424& \\
\hline
\end{tabular}
\caption{\label{tab:data} Thread counts over the four main sub-forums  (errata, 
exam, lecture and homework) of each course iteration, with their intervention 
ratio (I. Ratio), defined as the ratio of \# of intervened to non-intervened 
threads.}
\end{table} 


\section{Baseline (EDM'15)}
\label{sect:baseline}
The baseline system uses a maximum entropy classifier with the
following set of features: unigrams, thread forum type, student
affirmations to a previous post, thread properties (\# of posts,
comments, and posts+comments), average \# of comments per post, \# of
sentences in the thread, \# of URLs, and \# of timestamped references
to parts of a lecture video. The authors noted the imbalanced nature
of their datasets, with non-intervened threads greatly outnumbering
intervened ones. This motivated the use of class weights to
counterbalance the \# of non-intervened instances. Class weights, an
important parameter of this model, were estimated as the ratio of
negative to positive samples in the training instances.

\section{Discourse Feature Extraction}
\label{sect:feature}
We experimented with a prediction system based solely on
automatically-acquired discourse features from the PDTB-based discourse
parser from \cite{Lin2014}.
We employ this shallow discourse parser on the input to categorize identified discourse connectives
according to the PDTB tag set, and subsequently extract them for our use.
The parser first distinguishes discourse connectives (e.g., ``and'' can signal 
a discourse relation of `Expansion', but can also act as a coordinating conjunction). It then 
classifies them into one of several senses as specified by PDTB. PDTB 
categorizes the connectives into implicit and explicit connectives, each of 
which is assigned a sense. Senses are organized hierarchically, where the top 
Level--1 senses discriminate among 4 relations: 
`Contingency', `Expansion', `Comparison' and `Temporal'. 
We only used explicit connectives and Level--1 
senses as features. This is because \cite{Lin2014} report a low $F_1$ of 
39.6\% for extracting implicit connectives while that of explicit 
connectives is much better (86.7\%). Limiting to Level--1 senses also 
avoids sparsity issues. We found the distribution of the 4 Level--1 senses
as tagged by the discourse parser 
in our MOOC corpus was similar to that of the original PDTB annotated 
corpus built from 
newswire (see Table~\ref{tab:corpCompare}), supporting our decision to
use Level--1 senses. 

\begin{table}
\centering
\def\arraystretch{1.2}% 
\begin{tabular}{|l|r|r|r|r|}
\hline
\bf Corpus&  Exp.& Cont.& Comp. & Temp.\\
\hline
\small
\bf 14 MOOC corpus& 33\%&  28\%& 20\%& 19\%\\
\bf PDTB corpus&  34\%&  19\%& 29\%& 19\%\\
\hline
\end{tabular}
\caption{Distribution of PDTB level-1 senses for explicit connectives 
(top row) as tagged by the discourse parser, which are similar 
to those reported for the PDTB corpus \cite{Prasad2008}, bottom row.}
\label{tab:corpCompare}
\end{table}

We used  the Java version of the parser\footnote{Available at 
\url{https://github.com/WING-NUS/pdtb-parser}}, which comes pre-trained
with Sections~2--21 of the PDTB annotated corpus, using 
the {Level--1} relation senses. 
We note that although the discourse parser's performance on MOOC forum
text had not been previously evaluated, we decided to use the
pre-trained parser given that such PDTB discourse parsers have been
used to support a variety of downstream tasks using different corpora,
without retraining ({\it cf} Related Work). Additionally, re-training
is a resource intensive task, and we judged it to be a lower priority
to evaluate it specifically for MOOC data.

Each forum thread is treated as a document, where each post in the thread 
is treated as a paragraph. Since the parser identifies discourse relations 
within paragraphs of text, only within-post discourse relations 
were identified; this is appropriate as the parser was trained on
single-party narrative (newswire) rather than multiparty dialogue.  
We derive 25 features from the PDTB relation senses output by the parser. 
These constitute the discourse features identified as PDTB in 
Tables~\ref{tab:resultIn}, \ref{tab:resultOut} and \ref{tab:pdtb-robust}:

\begin{itemize}
\item Total number of all relation senses (1 feature): The sum of the 
frequencies (number of occurrences) of all four Level--1 senses;
\item Proportion of each sense (8 features): The absolute and relative 
frequency of each sense in a thread. Absolute frequency 
is normalized by thread length;
\item Proportion of sense sequences of length 2 (16 features): Normalized 
number of occurrences of each sense sequence of length 2 
(e.g., `Expansion'-`Contingency') in a thread divided by the total 
number of occurrences of all sense sequences of length 2.
\end{itemize}

\noindent We use the maximum entropy classifier with class weights 
as in the EDM '15 baseline for both PDTB and E+P systems. The implementations 
of the EDM '15 and the discourse based systems are 
publicly available\footnote{\url{https://github.com/WING-NUS/lib4moocdata}}.


\section{Evaluation}
\label{sect:result}
We evaluate the models under two evaluation schemes: an 
(i) in-domain scheme, and 
an (ii) out-of-domain scheme. We report the performance 
of the models in terms of precision (P), recall (R) and 
$F_1$ of the positive class.

The \textbf{in-domain setting} models were trained and evaluated separately on 
each MOOC using stratified five-fold cross validation. Stratification accounts 
for the highly imbalanced data and ensures that each fold had both positive and
negative samples. We see that the combined model E+P outperforms the baseline 
EDM'15~(Table~\ref{tab:resultIn}), on average. 

Drilling down, we see that while EDM'15 does well on the first 
offering (those with the ``-1'' suffix) of many courses
which have higher intervention ratios, E+P outperforms EDM'15 on subsequent 
offerings which typically have lower intervention ratios.
We also observe the $F_1$ scores for {\sc Classic-2} and {\sc Disaster-3} are 0.
Despite stratification, the \# of intervened threads per fold was 
too low for both courses ($\sim$3 to 4 per fold) due to their low intervention 
ratios (see Table \ref{tab:data}). As a result, both models are unable to 
predict any intervention for either course.
\begin{table}
\centering
\def\arraystretch{1.15}% 
\small
\begin{tabular}{|l|r|r|r||r|r|r|}
\hline 
&
\multicolumn{3}{c||}{EDM'15} & 
\multicolumn{3}{c|}{EDM'15 + PDTB}
\\
\hline
\bf Course&  P&  R& $F_1$& P&  R& $F_1$\\
\hline
{\sc Classic-1}& 25.0& 33.1& \textbf{28.5$^{**}$}& 22.7& 33.2& 27.0\\
\hline
{\sc Classic-2}& 0.0& 0.0& 0.0& 0.0& 0.0& 0.0\\
\hline
{\sc Reason-1}& 32.2& 48.2& \textbf{38.6}& 25.6& 41.8& 31.8\\
\hline
{\sc Reason-2}& 20.4& 47.5& 28.5& 27.3& 51.0& \textbf{35.5}\\
\hline
\hline
{\sc AccTalk}& 59.3& 44.7& \textbf{51.0}$^*$& 40.3& 50.7& 44.9\\
\hline
{\sc Clinical-1}& 49.7& 34.7& \textbf{40.8}& 51.0& 27.3& 35.6\\
\hline
{\sc Clinical-2}& 44.2& 61.3& 51.4& 49.0& 62.3& \textbf{54.9}\\
\hline
{\sc Disaster-1}& 14.7& 6.7& 9.2& 16.0& 9.4& \textbf{11.8$^{**}$}\\
\hline
{\sc Disaster-2}& 6.7& 5.0& \textbf{5.7}& 6.7& 3.3& 4.4\\
\hline
{\sc Disaster-3}& 0.0& 0.0& 0.0& 0.0& 0.0& 0.0\\
\hline
{\sc Nuclear-1}& 15.5& 16.8& \textbf{16.1}& 14.3& 12.9& 13.6\\
\hline
{\sc Nuclear-2}& 11.8& 19.4& 14.7& 20.0& 37.2& \textbf{26.0$^{**}$}\\
\hline
{\sc Nutrit-1}& 85.5& 57.8& \textbf{69.0$^{**}$}& 75.8& 58.2& 65.9\\
\hline
{\sc Nutrit-2}& 60.1& 9.5& 47.7& 61.3& 47.9& \textbf{53.8$^*$}\\
\hline
\hline
\textbf{Macro avg.}& 30.4& 29.6& 30.0& 29.3& \textbf{31.1}& \textbf{30.2}\\
\hline
\textbf{Weighted} & 37.3& 28.1& 32.0& 35.1& \textbf{30.0}& \textbf{32.4}\\
\textbf{macro avg.}& & & & & &\\
\hline
\end{tabular}
\caption{Model performance of EDM'15, with and without PDTB,
per MOOC, where each MOOC is evaluated individually 
(in-domain setting) using 5-fold stratified cross validation. 
Best performance is bolded; significance indicated where applicable 
($^* p < 0.05$; $^{**}p < 0.01$).}
\label{tab:resultIn}
\end{table}
In the \textbf{out-of-domain setting}, we use leave-one-out 
cross-course-validation (LOO-CCV) where models trained 
on 13 courses are tested on the 14\textsuperscript{th} unseen course. 
This evaluation setting more closely approximates the real world 
where universities hosting MOOCs have data from previously offered MOOCs and 
would want to train predictive models that could be deployed in upcoming 
courses. This evaluation shows which models are more robust when adapting 
to unseen out-of-domain data. Table~\ref{tab:resultOut} shows the 
performance of the EDM'15 and E+P models on each of the 14 MOOCs 
from LOO--CCV.

E+P betters EDM'15 performance by 3.4\% on average. The improved $F_1$ 
is largely due to a 5.7\% improvement in recall. We argue that, for the problem 
of intervention prediction, improving recall is more important than precision 
since missing an intervention is costlier than intervening on a less 
important thread. Here the performance outlier is the {\sc Clinical-1} MOOC, 
where EDM'15 performs significantly better than E+P, which may be partially 
attributed to the course having the smallest test set.

Further, Tables~\ref{tab:resultIn} and \ref{tab:resultOut} show that 
the E+P and EDM'15 models both benefit from access to more data in 
the out-of-domain setting. The EDM'15 model only improved by 2.7\%, 
whereas the E+P model improved by 5.7\%, showing the benefits 
of using domain-independent linguistic features to predict instructor 
intervention.
\begin{table}
\centering
\def\arraystretch{1.15}%
\small
\begin{tabular}{|l|r|r|r||r|r|r|}
\hline 
&
\multicolumn{3}{c||}{EDM'15} & 
\multicolumn{3}{c|}{EDM'15 + PDTB}
\\
\hline
\bf Course&  P&  R& $F_1$& P&  R& $F_1$\\
\hline
{\sc Classic-1} & 26.7& 3.1& 5.6& 23.6& 29.5& \textbf{26.2}$^{**}$\\
\hline
{\sc Classic-2} & 18.5& 31.3& 23.3& 18.2& 37.5& \textbf{24.5}\\
\hline
{\sc Reason-1} & 40.0& 12.3& 18.8& 50.0& 16.3& \textbf{24.6}\\
\hline
{\sc Reason-2} & 52.9& 29.0& 37.5& 35.6& 51.6& \textbf{42.1}$^{**}$\\
\hline
\hline
{\sc AccTalk} & 41.0& 26.7& 32.3& 50.0& 26.7& \textbf{34.8}\\
\hline
{\sc Clinical-1} & 81.8& 30.0& \textbf{43.9}$^*$& 71.4& 16.7& 27.0\\
\hline
{\sc Clinical-2} & 55.9& 76.0& 64.4& 59.2& 76.0& \textbf{66.7}\\
\hline
{\sc Disaster-1} & 25.6& 14.5& 18.5& 21.8& 25.0& \textbf{23.3}$^{**}$\\
\hline
{\sc Disaster-2} & 20.0& 4.8& \textbf{7.8}& 18.8& 4.8& 7.7 \\
\hline
{\sc Disaster-3} & 9.5& 11.1& 10.3& 8.6& 16.7& \textbf{11.3}$^{**}$\\
\hline
{\sc Nuclear-1} & 55.6& 4.8& 8.8& 66.7& 5.7& \textbf{10.6}\\
\hline
{\sc Nuclear-2} & 33.3& 15.6& \textbf{21.2}& 31.8& 15.6& 20.9\\
\hline
{\sc Nutrit-1} & 77.3& 62.4& \textbf{69.1}& 72.0& 63.4& 67.4\\
\hline
{\sc Nutrit-2} & 46.5& 52.4& 49.3& 54.2& 50.8& \textbf{52.5}$^*$\\
\hline
\hline
\textbf{Macro avg.}& 41.8& 26.7& 32.6& 41.6& \textbf{31.2}& \textbf{35.6}\\
\hline
\textbf{Weighted} & 42.7& 29.3& 34.7& 41.9& \textbf{35.0}& \textbf{38.1}\\
\textbf{macro avg.}& & & & & &\\
\hline
\end{tabular}
\caption{Prediction performance of EDM'15 and E+P systems in each of
  the 14 MOOCs where each one is evaluated (out-of-domain setting)
  using leave-one-out cross-course-validation (LOO--CCV). Best
  performance is bolded; significance indicated where applicable
  ($^* p < 0.05$; $^{**}p < 0.001$).}
\label{tab:resultOut}
\end{table}

\section{Discussion}
\label{sect:discuss}

To understand the observed performance of the PDTB-based features, we
probe further, answering two research questions that are natural
extensions of the results.\\

\noindent \textbf{RQ1.} {\it Are the PDTB features useful supplemental evidence, 
especially when simple features do not perform well?}

In each of the 5 courses where E+P performs better than EDM'15, the course 
iterations have smaller intervention ratios (see Tables~\ref{tab:data} and 
\ref{tab:resultIn}). For example, E+P betters EDM'15 on {\sc Clinical-2}, {\sc 
Reason-2} and {\sc Disaster-1} while EDM'15 has a better score on {\sc Clinical-1}, 
{\sc Reason-1} and {\sc Disaster-2}.
That is, PDTB features boost EDM'15 performance when there 
are fewer positive instances to learn from. This could be due to EDM'15's 
much larger feature space that requires more data to prevent sparsity. Note 
EDM'15 excluded stopwords, a subset of which are PDTB connectives, meaning 
that PDTB features contribute different information to the signal in the E+P 
model.  Our analysis of the contributions of features showed `Contingency' and 
`Expansion' relations to contribute the most. This may be due to their higher 
prevalence relative to the other discourse relations in the corpus.

Consider the example in Figure~\ref{fig:example-pdtbEDM}. E+P classifies 
this thread correctly while the EDM'15 model fails. This short thread does not 
contain many content words. In contrast, the discourse connectives in these 
student posts activate 16 of the 25 PDTB features.

\begin{figure}[h]
\small 
\begin{tabular}{|p{7.8cm}|}
\hline 

\textbf{Student 1 (Original poster)}:
Hi !!  I have a question about the 4th bar of the practice solution:
the V chord has three roots. Is that normal \textbf{or}$_{Exp}$ just a mistake?
Thank you. \\

\hline \\

\textbf{Student 2 (1st reply)}:
[Student1's name], it is not a mistake. \textbf{While}$_{Comp}$ not
as common \textbf{as}$_{Comp}$ merely doubling the root,
\textbf{if necessary}$_{Cont}$, you can triple the root. \textbf{As}$_{Cont}$
you see in this case, the root is tripled to smooth out the voice leading.
\textbf{Otherwise}$_{Comp}$, the tenor would be on the E, \textbf{and}$_{Exp}$
that would (probably) \textbf{result in}$_{Cont}$ parallel fifths moving to beat 4. \\

\hline \\

\textbf{Student 3}: Thanks [Student2's name] \textbf{but}$_{Comp}$, are you sure?
I don't see it clear.... \\

\hline \\

\textbf{Instructor's reply}: Hi [Student3's name]. Please read these threads. 
It has been discussed before. 
Thanks. $<$URL$>$ ... \\

\hline
\end{tabular}
\caption{Both PDTB and E+P capture this intervention, while 
EDM'15 fails to capture this clarifying intervention.}
\label{fig:example-pdtbEDM}
\end{figure}

While the use of explicit discourse connectives helped the classification 
task in many cases (e.g., Figure~\ref{fig:example-pdtbEDM} and 
\ref{fig:example-pdtb}), the PDTB parser does not cover all of 
the observed discourse connectives or their expressed senses. Examples of 
connectives (in bold) that were not recognized include:\\

\indent -- ...not \textbf{as nice as} I thought it would be... \\
\indent -- \textit{ There's \textbf{only so much} melodic expressiveness...\\ 
\indent \hspace{3mm} when using \textbf{nothing but} chord tones...}\\

\noindent The presence of these connectives in our data is consistent with recent 
calls \cite{forbesriley2016} to modify the PDTB relation inventory by 
adding a broader set of tags, such as those suggested by \citeauthor{tonelli2010} 
(\citeyear{tonelli2010}). Increasing the PDTB's coverage of explicit connectives 
would likely improve results.
The added use of implicit connectives may also 
improve prediction performance, should 
implicit connective detection and classification be improved. 
Figure~\ref{fig:example-pdtb} shows an example where implicit connectives may 
strengthen signals from discourse features. We also note that there are cases, 
such as in Figure~\ref{fig:example-danger}, which lack discourse and lexical 
signals. The intervention here is instead triggered by domain knowledge. These 
excerpts exemplify the difficulty of our prediction task.\\

\begin{figure}[t]
\small 
\begin{tabular}{|p{7.8cm}|}
\hline 

\textbf{Student (Original poster)}: Try to search Epipen auto-injector. 
It is an epinephrine single-dose injection used to aid in severe 
allergic reaction or anaphylaxis. It's expiration date 
is around 1 year \textbf{after}$_{Temp}$ manufacturing date. \\

\hline \\

\textbf{Instructor's reply}: Be very careful concerning epipens. They are for 
severe reactions only ... They are prescription only for very good reasons as 
they affect the heart... It is probably bad for you ... \\

\hline

\end{tabular}
\caption{An instructor intervention to correct a student's misconception. 
Both the EDM'15 and E+P systems fail to predict this intervention.  
} 
\label{fig:example-danger}
\end{figure}

\begin{figure}[t]
\small 
\begin{tabular}{|p{7.8cm}|}
\hline 
\textbf{Student 1 (Original Poster)}: Well, hurricanes are \#1 in the summer time. 
/\textbf{So}/$_{Cont}$ I always have a hurricane kit handy 
-3 days worth of supplies. 
/\textbf{Also}/$_{Exp}$,Floods, especially coastal and flash flooding are 
prevalent. Tornadoes, severe storms with damaging winds and hail. 
/\textbf{But}/$_{Comp}$ We don't have to worry about 
snow \textbf{or}$_{Exp}$ fires too much. Earthquakes, either, 
\textbf{although}$_{Comp}$ we do have a fault line 
nearby. /\textbf{so far}/ We've had a couple of very minute tremors 
in my lifetime, \textbf{but}$_{Exp}$ nothing that is really noticed.  \\

\hline \\

\textbf{Student 2}: Tornadoes and hail/wind storms are the most prevalent 
disaster to my area, \textbf{although}$_{Comp}$ we have been hit with just 
about everything, including a hurricane (Hurricane Ike took our roof)! 
Flooding is an issue in this area, \textbf{as}$_{Exp}$ is extreme winter 
weather on occasion; in recent years, significant snowfall (2 feet) in a short 
period of time basically paralyzes communities such \textbf{as}$_{Comp}$ ours 
that do not contend with such very often. We have \textbf{also}$_{Exp}$ had 
significant ice storms that have caused incredible damage. We are well aware 
of the fact that we could get a significant earthquake, 
\textbf{though}$_{Comp}$ fortunately we have 
had only minor issues in that regard...  \\

\hline \\

\textbf{Instructor's reply}: I was involved in some of the response
after Katrina...  I met some amazing folks and saw some real
devastation. One thing I never got used to ... Still I was glad to be
there and the people - amazing ...  \\

\hline
\end{tabular}
\caption{An instructor intervention to build common ground with students. The 
PDTB model predicts this intervention, while EDM'15 and E+P fail. Also shown 
in the first post are \textit{implicit} connectives, within /../, that are not
captured.
}
\label{fig:example-pdtb}
\end{figure}

\begin{table}
\centering
\def\arraystretch{1.15}% 
\small
\begin{tabular}{|l|r|r|r||r|r|r|}
\hline 
&
\multicolumn{3}{c||}{EDM'15} & 
\multicolumn{3}{c|}{PDTB}
\\
\hline
\bf Course&  in&  out & gain & in& out & gain\\
\hline
{\sc Classic-1} & 28.5& 5.6& -22.9& 22.9& 32.5& 9.5\\
\hline
{\sc Classic-2} & 0.0& 23.3& 23.3& 14.5& 21.9& 7.4\\
\hline
{\sc Reason-1} & 38.6& 18.8& -19.8& 12.2& 30.3& 18.1\\
\hline
{\sc Reason-2} & 28.5& 37.5& 9.0& 21.9& 33.7& 11.8\\
\hline
\hline
{\sc AccTalk} & 51.0& 32.3& -18.7& 32.9& 31.1& -1.8\\
\hline
{\sc Clinical-1} & 40.8& 43.9& 3.1& 6.2& 32.6& 26.4\\
\hline
{\sc Clinical-2} & 51.4& 64.4& 13.1& 20.6& 20.5& -0.1\\
\hline
{\sc Disaster-1} & 9.2& 18.5& 9.3& 8.9& 4.0& -4.8\\
\hline
{\sc Disaster-2} & 5.7& 7.8& 2.1&14.7 & 16.1& 1.4\\
\hline
{\sc Disaster-3} & 0.0& 10.3& 10.3&2.0 & 7.1& 5.2\\
\hline
{\sc Nuclear-1} & 16.1& 8.8& -7.3& 10.8& 23.9& 13.0\\
\hline
{\sc Nuclear-2} & 14.7& 21.2& 6.5& 21.7& 20.3& -1.4\\
\hline
{\sc Nutrit-1} & 69.0& 69.1& 0.1& 6.7& 9.9& 3.18\\
\hline
{\sc Nutrit-2} & 47.7& 49.3& 1.6& 9.7& 14.8& 5.11\\
\hline
\hline
\textbf{Macro avg.}& 30.0& 32.6& \textbf{2.6}& 16.4& 25.3& \textbf{8.9}\\
\hline
\textbf{Weighted} & 32.0& 34.7& \textbf{2.7}& 11.8& 23.4& \textbf{11.6}\\
\textbf{macro avg.}& & & & & &\\
\hline
\end{tabular}
\caption{Prediction performance of EDM'15 and PDTB 
systems for each of the 14 MOOCs in the (in)-domain 
and (out)-of-domain evaluations.}

\label{tab:pdtb-robust}
\end{table}

\noindent \textbf{RQ2.} {\it Are PDTB features more robust than 
vocabulary-based  features?}

Consider the performance differences of the EDM'15 and PDTB models 
between the in-domain and out-of-domain evaluation 
settings~(see Table~\ref{tab:pdtb-robust}). 
Using PDTB features results in an average improvement of 11.6\% when 
evaluating models out-of-domain, whereas EDM'15 only improves by 2.7\%.
EDM'15 performance drops greatly on {\sc Classic-1}, {\sc Reason-1}, and {\sc 
Acctalk} due to the out-of-domain data; in contrast, PDTB gains on 
10 of the 14 courses. In the example in Figure~\ref{fig:example-pdtb}, EDM'15 
and E+P fail while PDTB predicts correctly. Frequent words from this course 
(e.g., ``tornado'', ``earthquake'') are rare across MOOCs; this weakens the 
EDM'15 model. 

These findings suggest that PDTB may result in further gains were data added 
from more courses, while that of the vocabulary-based EDM'15 model may worsen or 
not scale. Similarly, Chandrasekaran et al. 
(\citeyear{chandrasekaran2015learning}) 
did not see improvements in the EDM'15 model when they went from a 13 course 
training set to 60 courses. This demonstrated lack of robustness in the EDM'15
model is not surprising given its abundant use of domain-specific vocabulary. 
There were considerably more out-of-domain unigram features (76,382) than 
in-domain unigram features (15,161, on average). This steep increase 
in feature space and resulting sparsity hampers the EDM'15 model's ability to 
benefit from a scaled corpus. In contrast, both in-domain and out-of-domain 
versions of the PDTB model have the same number of features.

\textbf{Potential Improvements.} The above results indicate that performance 
improvements due to PDTB features scale better than vocabulary-based features.
To harness similar improvements from large scale data, the performance and 
robustness of the discourse parser needs to be improved. Current limitations 
of discourse parsers (e.g., their inability to process equations and symbols, 
lack of near real-time output) inhibit scaling the predictive power of PDTB 
and E+P models. Discourse parser improvements would therefore enable the 
development and use of better predictive models.

\section{Conclusion}
\label{sect:conclusion}
In this study, we better the state-of-the-art for intervention prediction 
by augmenting it with PDTB relation based features. Further, on select MOOC 
offerings PDTB relations alone performed comparably to the state-of-the-art.
Unlike vocabulary based models, PDTB based features were shown to be robust 
to domain differences across MOOCs. This domain independence supports the 
improved prediction of instructor interventions. The current $F_1$ scores are 
still markedly low. Better modeling of the instructor may help boost performance. 
We plan to tackle this in two ways.  First, we will model instructor intervention 
based on the threads they have seen because they cannot intervene in threads that 
they have not seen. Second, we will model intervention based on the role that 
different types of instructional staff play. We expect teaching assistants, course 
alumni (also known as ``community TAs'' and ``mentors''), and faculty to have 
different motivations for intervening. They may also dedicate different amounts 
of time to course forums. As a result, modelling these factors or individual 
instructor preferences could improve prediction performance. 

\section*{Acknowledgments}
This research is funded in part by NUS Learning Innovation Fund --
Technology grant \#C-252-000-123-001, and by the Singapore National
Research Foundation under its International Research Centre @
Singapore Funding Initiative and administered by the IDM Programme
Office, and by an NUS Shaw Visiting Professor Award. The research is also 
funded in part by the Learning Research and Development Center at the 
University of Pittsburgh and by a Google Faculty Research Award. We would like 
to thank the University of Pittsburgh's Center for Teaching and Learning, for 
sharing their MOOC data. 

\bibliography{2aaai}
\bibliographystyle{aaai}

\end{document}